\documentclass[12pt,a4paper]{article}
\usepackage[utf8]{inputenc}
\usepackage{graphicx}
\usepackage{amsmath, amssymb}
\usepackage{siunitx}
\usepackage{mhchem}
\usepackage{booktabs}
\usepackage{geometry}
\geometry{margin=2.5cm}

\title{TP3 – Détermination d'une constante de partage}
\author{}
\date{}

\begin{document}
	\maketitle
	
\section{Objectifs}

Rappeler brièvement le but de l'expérience : extraction de l'acide propanoïque, titrage, constante de partage. L’équilibre de partage s’écrit :
\[ \ce{C2H5COOH (aq) <=> C2H5COOH (org)} \]

	% Commentaire : Rédigez ici avec vos mots l’objectif du TP, en insistant sur l’intérêt pratique de l’extraction liquide-liquide.

	
\section{Résultats expérimentaux}
	\begin{itemize}
		\item Volume de solution titrée : \SI{}{\milli\liter}
		\item Concentration de la soude : \SI{}{\mole\per\liter}
		\item Volume équivalent mesuré : \SI{}{\milli\liter}
	\end{itemize}
	
	\section{Exploitation}
	\subsection{Constante de partage $K$}
	Calcul détaillé à partir des données brutes.
	
	\subsection{Rendement d’extraction}
	Définition, calcul et interprétation.
	
	\section{Analyse des incertitudes}
Lister les principales sources d’incertitude et indiquer la valeur numérique retenue pour chacune :
	\begin{itemize}
		\item Pipette jaugée : volume nominal $V = \SI{10,0}{\milli\liter}$, incertitude $\pm \SI{0,03}{\milli\liter}$.
		\item Burette graduée : volume équivalent lu $V_\text{eq}$, incertitude de lecture $\pm \SI{0,05}{\milli\liter}$.
		\item Fiole jaugée : volume $\SI{100,0}{\milli\liter}$, incertitude $\pm \SI{0,05}{\milli\liter}$.
		\item Concentration de la soude $C_\text{NaOH}$ : incertitude relative $\approx 1\,\%$.
		\item Appréciation du virage coloré : erreur de lecture subjective estimée à $\pm \SI{0,05}{\milli\liter}$.
	\end{itemize}
	
	\section{Simulation Monte Carlo}
\subsection{Paramètres retenus}
Pour chaque grandeur, on associe une loi de probabilité :
\begin{itemize}
	\item Pipette jaugée : loi uniforme sur $[V - 0,03,\, V + 0,03]$.
	\item Burette : loi uniforme sur $[V_\text{eq} - 0,05,\, V_\text{eq} + 0,05]$.
	\item Fiole jaugée : loi uniforme sur $[100,0 - 0,05,\, 100,0 + 0,05]$.
	\item Concentration de NaOH : loi normale centrée sur la valeur mesurée avec écart-type $= 1\% \times C_\text{NaOH}$.
\end{itemize}

\subsection{Consignes de simulation}
\begin{enumerate}
	\item Générer aléatoirement $N = 10\,000$ jeux de données selon les lois ci-dessus.
	\item Calculer la constante de partage $K$ pour chaque jeu de données.
	\item Déterminer la moyenne, l’écart-type, et l'incertitude pour un niveau de confiance de 95 \%.
\end{enumerate}

\subsection{Résultats obtenus}

	\section{Résultats collectifs}
	Tableau récapitulatif des résultats des 12 groupes, statistiques (moyenne, écart-type, intervalle de confiance).
	
	\section{Discussion et conclusion}
	- Comparaison extraction simple vs. multiple.  
	- Discussion sur l’écart simulation/expérience.  
	
\end{document}
